\chapter{Related Work}
\todo{}

Fortress. Math notation. Large set of operator symbols (everything in unicode, more or less). Syntax extension seems to be a tangent, not implemented?

Intentional whatever. MPS. 

Structure editors in general (CPS, etc.) YALE\cite{zavodnik}: similar approach, but a completely different set of goals---speed compilation by avoiding lexing and syntactic analysis; prevent errors. Similarly to CPS, leaves expressions to be entered as text (because they are relatively simple to parse).

Visual programming languages. 

barista(eisenberg)

sikuli - embeds screenshots in a python program (more an example than anything)

XMLisp, sorta

\todo{more...}



%\section{JUNK: Future Work}

%\temp{more checking: scope? loop/recur? type-check the reductions?}

%\temp{:program for statements, blocks, etc.}

%\temp{source location and debugging/stack traces, etc.}

%\temp{toolchain: diff, what else?}

%\temp{integrate with an IDE}


%\subsection{Editing}

%\todo{put some of this into examples...}

%\temp{result: so-so. left/right vs. up/down is confusing. would it be better to distinguish between left/right and up/down and use some other keystrokes for parent/child?}

%\temp{goals: clarify tree structure; correspond to visual layout; minimize keystrokes; be reversible}

%\temp{Is this discussion needed? here?} This simple model is sufficient but sometimes it requires several actions to change the selection to the desired node, and sometimes a sequence of actions produces an unexpected result. For example, once the first visible child of a node is selected, the \emp{select previous sibling} action is ineffective, so that action cannot be used to traverse beyond the current parent. A related action might be \emp{select previous cousin}, which would select either the previous sibling, or no such sibling exists, then the last visible child of the parent's previous sibling. Alternatively, \emp{select previous ???} would move from the first visible sibling to the \emp{parent's} previous sibling.

%\todo{talk about ways of extending the notion of selection to more than a single node?}

