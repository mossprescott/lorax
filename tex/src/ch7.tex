\chapter{Conclusion}

In this thesis, I present tools for defining languages defined in terms of a grammar for ASTs and reduction to a general-purpose presentation language, and a structure editor for any such language. The reductions are written in a new meta-language which is defined in the same way, from syntax extension of a minimal kernel language. The editor provides superior readability of simple programs, plus accurate rendering of mathematical notation. It also eliminates some of the work of entering and modifying programs, offsetting the potential downsides of the structure editing approach.

Moreover, by reducing the difficulty of implementing syntax extensions, and extending the scope of what extensions can do, the system makes the creation of a new language as an extension of a kernel language much simpler and more powerful than before. Much of the work to render and provide editing for language elements is done in a general way, via the presentation language. New languages and new constructs can be specified quite economically as reductions to this language, which provides primitives for the commonly-used visual elements, and can easily be extended with additional symbols.
