%
% Macros and definitions:
%

%%% override thesis.cls's definitions for floats (figures):
\setcounter{topnumber}{2} % 1
\renewcommand*{\topfraction}{.9} % .4
\setcounter{totalnumber}{4} % 1?
\renewcommand*{\textfraction}{.07}
\renewcommand*{\floatpagefraction}{.7} % .3
%\setcounter{dbltopnumber}{1}
%\renewcommand*{\dbltopfraction}{.4}
%\renewcommand*{\dblfloatpagefraction}{.3}
\setlength{\abovecaptionskip}{5pt}  % 10pt
\setlength{\belowcaptionskip}{5pt} % 10pt


\def\Meta{Lorax}

\def\emp{\textit}  % thesis.cls changes \emph to bold, which sucks, so I define my own

\newcommand{\temp}[1]{ {\color{blue}#1} }

% font for literals in the presentation:
\newcommand{\keyword}[1]{\textbf{#1}}%{\ensuremath{\mathtt{#1}}}
\newcommand{\clojure}[1]{\texttt{#1}}%{\ensuremath{\mathtt{#1}}}

\def\lsyn{ {\color{blue}\left(\!\right.} }
\def\rsyn{ {\color{blue}\left.\!\right)} }

\newcommand{\todo}[1]{ {\textit{\color{red}[TODO: #1]}} }


%
% commands for reduction equations:
%
\newcommand{\mapnode}[2]{ {\mathbf{#1}~\{ #2 \}} }
\newcommand{\attr}[2]{ {\textit{#1} \mapsto #2} }


\newcommand{\seqnode}[2]{ {\mathbf{#1}~[ #2 ]} }

\newcommand{\reduces}{ \quad \longrightarrow \quad }

% \TeX seems to be built-in
%\def\tex{T\kern-.1667em\lower.5ex\hbox{E}\kern-.125emX\spacefactor1000}


% a way of setting off ???
%\newcommand{\syntax}[1]{\ensuremath{\fbox{\ensuremath{#1}}}}

% 
%\definecolor{literal}{rgb}{.8,.8,.6}
%\definecolor{literalbg}{rgb}{1,1,.8}

% deprecated:
%\newcommand{\embed}[1]{\ensuremath{\fcolorbox{literal}{literalbg}{\ensuremath{#1}}}}
%\newcommand{\unbed}[1]{\ensuremath{\fcolorbox{literal}{white}{\ensuremath{#1}}}}

%\newcommand{\quoted}[1]{\ensuremath{\fcolorbox{literal}{literalbg}{\ensuremath{#1}}}}
%\newcommand{\unquoted}[1]{\ensuremath{\fcolorbox{literal}{white}{\ensuremath{#1}}}}

%\newcommand{\node}[2]{\ensuremath{#1 \: \{ #2 \}}}
%\newcommand{\attr}[2]{\ensuremath{\mathrm{#1} : \: #2}}

%\newcommand{\reducep}{\ensuremath{\overset{present}{\longrightarrow}}}
%\newcommand{\reducec}{\ensuremath{\overset{compile}{\longrightarrow}}}

%\newcommand{\gray}[1]{ {\color[gray]{.7}#1}}
%\newcommand{\gparens}[1]{ \gray{(} #1 \gray{)} }

% formatting for selections:
%\definecolor{selection}{rgb}{1,.5,.8}
