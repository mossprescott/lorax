\section{Introduction}

\begin{quote}
``Programs are not text; they are hierarchical compositions of computational structures and should be edited, executed, and debugged in an environment that consistently acknowledges and reinforces this viewpoint.''\cite{teitelbaum}
\end{quote}

Those are the opening words of the... and they are as true today as they were in 1981. However, the Cornell Program Synthesizer failed to spark the revolution its creators might have intended.

\temp{the AST is the program}

\temp{programmer and machine should collaborate to create it}

Make the structure of the program explicit, but derive a human-readable presentation from it. Like a WYSIWYG editor (word processor, SVG). 

If you stop there, you have a \emph{structure editor}\cite{teitelbaum}. This isn't compelling, because advantages don't outweigh the inconveniences. So go further by supporting the introduction of new constructs.

Claims:
\begin{enumerate}
\item Using an \emph{abstract syntax tree} as the ultimate source code can make programs easier to read and understand, and expands scope of language.
\item A meta-programming approach makes it a viable alternative.
\item My prototype proves that [??]
\end{enumerate}


\subsection{Related Work}
Fortress. 

Intentional Whatever. MPS. 

Structured editors in general (CPS, etc.) Visual programming languages. 

barista(eisenberg)

XMLisp, sorta

\todo{more...}

