The programmer's task is to construct the abstract syntax tree (AST) which best represents the program she wishes to run. Ideally, she would do so by first building a language that easily expresses the program, then constructing an AST in that new language. Unfortunately, in a conventional language the programmer manipulates the AST only indirectly, through a parser. This makes the language difficult to extend and constrains the visualization of the program. In other words, what she can write and read is dictated by the input the parser is able to handle.

I explore an alternative approach: represent source code directly as an AST and derive both an executable program and a readable presentation from it. I present a flexible representation for ASTs, a general mechanism for transforming these trees, and a language for grammars which allows concrete syntax and semantics to be defined via these transformations. I show that this approach is modular, easy to understand, and expressive enough to define novel syntax and semantics.

My prototype system, \Meta, demonstrates the new approach. Reductions for presentation and execution are written in a new language based on a functional kernel language with meta-programming features. Syntax is not limited to simple text but may include richer notation for easier reading and understanding. This rich syntax is rendered using some of the algorithms of \TeX. I assembled these components and others into a structure editor. In \Meta, the barrier to entry for the creation of languages is lowered, making it practical for programmers to express solutions in the terms and the notation which are closest to the problem domain.