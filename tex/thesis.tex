\documentclass[12pt]{amsart}

\usepackage{color}
\usepackage{geometry}                % See geometry.pdf to learn the layout options. There are lots.
\geometry{letterpaper}                   % ... or a4paper or a5paper or ... 
%\geometry{landscape}                % Activate for for rotated page geometry
\usepackage[parfill]{parskip}    % Activate to begin paragraphs with an empty line rather than an indent
%\usepackage{graphicx}
%\usepackage{amssymb}
%\usepackage{epstopdf}
%\DeclareGraphicsRule{.tif}{png}{.png}{`convert #1 `dirname #1`/`basename #1 .tif`.png}

\title{Building a Language Builder}
\author{Moss Prescott}
\date{18 August 2010}

%
% Syntax macros:
%

\newcommand{\todo}[1]{ {\textit{\color{red}[TODO: #1]}} }

% font for literals in the presentation:
\newcommand{\keyword}[1]{\texttt{#1}}%{\ensuremath{\mathtt{#1}}}
\newcommand{\clojure}[1]{\texttt{#1}}%{\ensuremath{\mathtt{#1}}}

% a way of setting off ???
%\newcommand{\syntax}[1]{\ensuremath{\fbox{\ensuremath{#1}}}}

% 
%\definecolor{literal}{rgb}{.8,.8,.6}
%\definecolor{literalbg}{rgb}{1,1,.8}

% deprecated:
%\newcommand{\embed}[1]{\ensuremath{\fcolorbox{literal}{literalbg}{\ensuremath{#1}}}}
%\newcommand{\unbed}[1]{\ensuremath{\fcolorbox{literal}{white}{\ensuremath{#1}}}}

%\newcommand{\quoted}[1]{\ensuremath{\fcolorbox{literal}{literalbg}{\ensuremath{#1}}}}
%\newcommand{\unquoted}[1]{\ensuremath{\fcolorbox{literal}{white}{\ensuremath{#1}}}}

%\newcommand{\node}[2]{\ensuremath{#1 \: \{ #2 \}}}
%\newcommand{\attr}[2]{\ensuremath{\mathrm{#1} : \: #2}}

%\newcommand{\reducep}{\ensuremath{\overset{present}{\longrightarrow}}}
%\newcommand{\reducec}{\ensuremath{\overset{compile}{\longrightarrow}}}

%\newcommand{\gray}[1]{ {\color[gray]{.7}#1}}
%\newcommand{\gparens}[1]{ \gray{(} #1 \gray{)} }

% formatting for selections:
%\definecolor{selection}{rgb}{1,.5,.8}


\begin{document}

\begin{abstract}
\todo{}
\end{abstract}

% Follows the abstract (but appears first), in amsart, for some reason:
\maketitle


%\newpage
\tableofcontents

\newpage

% Jeremy:
% - step back and remember the big picture
% - keep separate contributions separate: make sure each idea stands on its own, so you could use one without buying into the other.

% How would you build a language with incompatible semantics (e.g. Java)? On this kernel language? No, instead reduce to a different kernel, and execute that...
% Ex. Python, with extensions, and generate Python? Or Python AST?


% alt. structure:
%
% foundations
% - nodes, values, refs
% - specs
% - reductions
% - characteristics
%   - self-describing
%   - source locations
%   - diff
% - compared to other approaches
%   - XML
%   - ML/Haskell ADT
%   - s-expressions
%
% building languages
% - abstract node types and instances
% - names
% - expr. language (one or more)
% - kernel language (one or more)
%   - meta-programming
% - grammars
% - foreign languages
%
% implementation
% - platform/kernel language
%   - clojure, compilation in general, quotations
% - expr. language (the one I implemented)
%   - principles
%   - :expr, and the fancy shit it provides
%   - :view
%   - reducing :expr to :view
% - names
% - parens/"precedence"
% - editing
%
% future work

\input ch1.tex  % Introduction
\input ch2.tex  % Foundations
\input ch3.tex  % Implementation
\input ch4.tex  % Presentation
\input ch5.tex  % Editing

\section{Future Work}

% references


\end{document}
