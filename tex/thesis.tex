\documentclass[12pt]{amsart}

\usepackage{color}
\usepackage{geometry}                % See geometry.pdf to learn the layout options. There are lots.
\geometry{letterpaper}                   % ... or a4paper or a5paper or ... 
%\geometry{landscape}                % Activate for for rotated page geometry
\usepackage[parfill]{parskip}    % Activate to begin paragraphs with an empty line rather than an indent
\usepackage{graphicx}
%\usepackage{amssymb}
%\usepackage{epstopdf}
%\DeclareGraphicsRule{.tif}{png}{.png}{`convert #1 `dirname #1`/`basename #1 .tif`.png}
\DeclareGraphicsRule{.tiff}{png}{.png}{`convert #1 `dirname #1`/`basename #1 .tiff`.png}

\input src/defs.tex

\title{\Meta: a Toolkit for Language Construction}
\author{Moss Prescott}
\date{18 August 2010...}

\setcounter{tocdepth}{3}

\begin{document}

\begin{abstract}
\input src/abstract.tex
\end{abstract}

% Follows the abstract (but appears first), in amsart, for some reason:
\maketitle


%\newpage
\tableofcontents

%\newpage

% Jeremy:
% - step back and remember the big picture
% - keep separate contributions separate: make sure each idea stands on its own, so you could use one without buying into the other.

% How would you build a language with incompatible semantics (e.g. Java)? On this kernel language? No, instead reduce to a different kernel, and execute that...
% Ex. Python, with extensions, and generate Python? Or Python AST?


% Some contributions are not hilited very well:
% - parenthesization
% - 



% alt. structure:
%
% foundations
% - nodes, values, refs
% - specs
% - reductions
% - characteristics
%   - self-describing
%   - source locations
%   - diff
% - compared to other approaches
%   - XML
%   - ML/Haskell ADT
%   - s-expressions
%
% building languages
% - abstract node types and instances
% - names
% - expr. language (one or more)
% - kernel language (one or more)
%   - meta-programming
% - grammars
% - foreign languages
%
% implementation
% - platform/kernel language
%   - clojure, compilation in general, quotations
% - expr. language (the one I implemented)
%   - principles
%   - :expr, and the fancy shit it provides
%   - :view
%   - reducing :expr to :view
% - names
% - parens/"precedence"
% - editing
%
% future work

\input src/ch1.tex  % Introduction
\input src/ch2.tex  % Foundations
\input src/ch3.tex  % Languages
\input src/ch4.tex  % Prototype
\input src/ch5.tex  % Examples
\input src/ch6.tex  % Future Work

% references


\end{document}
